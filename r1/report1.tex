\documentclass[11pt, a4paper]{article}

\usepackage{jheppub}
\bibliographystyle{JHEP.bst}
\allowdisplaybreaks

\usepackage{hyperref}
\usepackage{amsmath,amsfonts,amssymb,amsthm,natbib,graphicx,mathtools,mathrsfs}
\usepackage{braket}
\usepackage{enumitem}
\usepackage{slashed}
\usepackage{relsize}
\usepackage{bm}

\begin{document}

\title{The chiral anomaly in Weyl semimetals}
\author{K. Okyere}

\abstract{\dots}

\maketitle

\section{Introduction}
The Schrödinger equation is not Lorentz invariant thus, it violates special relativity.
However, a relativistic quantum mechanics can be achieved by
quantising the relativistic dispersion relation. This yields the Klein-Gordon
equation~\cite{Klein} whose half-iterate is the famous Dirac equation~\cite{Dirac:102829}
\begin{equation}
  (i\hbar\gamma^\mu\partial_\mu - mc)\psi = 0.
\end{equation}
In the Weyl basis, the massless Dirac equation simplifies to two independent
Weyl equations~\cite{Weyl} of opposite chirality
\begin{equation}
  i\hbar\sigma^\mu\partial_\mu \psi_1 = 0, \quad
  i\hbar\bar{\sigma}^\mu\partial_\mu\psi_2 =0;
\end{equation}
\begin{equation}
  \psi = \begin{pmatrix} \psi_1 \\ \psi_2  \end{pmatrix}, \quad 
  \gamma^\mu = \begin{pmatrix} 0 & \sigma^\mu \\ \bar{\sigma}^\mu & 0 \end{pmatrix}.
\end{equation}
The massless Dirac equation is of great interest in quantum field theory because
Noether's theorem necessitates conservation of the associated axial current
in the classical theory; however, this is violated by
the chiral anomaly~\cite{Adler,JackiwBell}
\begin{equation}
  {\partial_\mu j^\mu_5 }|_{m=0} \neq 0.
\end{equation}
This result is deeply connected to the topology of the quantum field theory
(c.f.~\cite{KofiEthan} for a review), and 
as we shall see, an analogous result in condensed matter systems is also
profoundly topological.

\section{Electronic structure and Berry phase}

Insulators are said to be conventional (topologically trivial) if a
continuous transformation (homeomorphism) of lattice parameters can
transform the band structure into the band structure of the vacuum;
the topology is the same as the ``atomic limit''. However, not all 
insulators are topologically trivial.
Suppose a system has an initial Hamiltonian, with a discrete non-degenerate 
energy spectrum, that evolves
continuously as $\widehat{H}(\lambda^\mu)$, where $\lambda^\mu(t)$ are time-dependent parameters.
Furthermore, let $\ket{n(\lambda^\mu(t))}$ be instantaneous eigenstates 
corresponding to energies $E_n(t)$. Assuming that the Hamiltonian evolves slowly 
so that $\braket{m|\partial_t \widehat{H}|n} \approx 0$ (the adiabatic approximation),
then an initial eigenstate evolves as
\begin{equation}
  \ket{n(t = t_0)} \xrightarrow{t \rightarrow t_1} e^{i \theta_n(t_1) +i \phi_n(t_1)}\ket{n(t_0)},
\end{equation}
where $\theta_n(t)$ and $\phi_n(t)$ are the dynamical and Berry phases respectively
\begin{equation}
\theta_n(t_1) = -\frac{1}{\hbar}\int_{t_0}^{t_1}{E_n(\tau)\,d\tau},
\end{equation}
\begin{equation}
  \phi_n(t_1) = i\int_{t_0}^{t_1}{\braket{n |\partial_{\mu}| n}
  \frac{d\lambda^\mu}{d\tau} \,d\tau}
  = i\int_C{ \braket{n |\partial_\mu| n}\,d\lambda^\mu };
  \qquad \partial_\mu \coloneq \frac{\partial}{\partial \lambda^\mu}
\end{equation}
This is the adiabatic theorem~\cite{AdiabaticThrm,Berry},
and the last path integral describes how the phase of eigenstates changes as
the system evolves through parameter space.
We now define the Berry potential/connection as
\begin{equation}
  A_\mu(\lambda) =  \braket{n |\partial_\mu| n}.
\end{equation}
In quantum mechanics, the phase of the eigenstates is unphysical,
the observable quantities are phase differences;
hence, changing the phase of $\ket{n}$ should leave the equations of motion unchanged.
Subsequently, the system is invariant under the transformation
\begin{equation}
  \ket{n} \rightarrow e^{i\beta(\lambda)}\ket{n(\lambda)}
  \implies A_\mu \rightarrow A_\mu(\lambda) - \partial_\mu \beta(\lambda).
\end{equation}
This redundancy is similar to gauge freedom in electromagnetism and
leaves the Berry phase invariant if the path through parameter space is a closed path.
Hence we can construct a $\mathrm{U(1)}$ gauge theory for cyclic variations of
$\widehat{H}(\lambda^\mu)$ analogous to electromagnetism
where the Berry potential is the gauge connection and the corresponding
field strength
\begin{equation}
  \Omega_{\mu\nu} = \partial_\mu A_\nu - \partial_\nu A_\mu,
\end{equation}
is called the Berry curvature.
This is a gauge invariant quantity thus, using Stokes' theorem~\cite{StokesThrm},
we can construct another expression for the Berry phase
\begin{equation}
\phi_n = i \oint_C{A_\mu\,d\lambda^\mu} = i \int_S{\Omega_{\mu\nu}\,ds^\mu\wedge ds^\nu},
\end{equation}
where C is the boundary of the surface S in parameter space. The last integral is the integral of the
curvature form of a vector bundle on a smooth manifold and by using the Chern-Weil theory
we have that
\begin{equation}
  \phi_n = 2\pi c_1,
\end{equation}
where $c_1$ is the first Chern class which is a topologically
invariant integer~\cite{Chern},
also known as the TKNN invariant because of its relation to
the quantum Hall effect~\cite{TKNN}.
Bloch's theorem states that the energy eigenstates of the
$n^{th}$ band of a perfect crystal are of the form
\begin{equation}
  \psi_{n\bm{k}}(\bm{r}) = e^{i\bm{k}\cdot\bm{r}}u_{n\bm{k}}(\bm{r}),
\end{equation}
where $u_{n\bm{k}}$ has the periodicity of the direct lattice~\cite{Bloch}.
Thus, the Berry potential becomes 
\begin{equation}
  A_{n\mu}(\bm{k}) =  \braket{u_{n\bm{k}} |\partial_\mu| u_{n\bm{k}}},
\end{equation}
and for two dimensional lattices we have a topological invariant
\begin{equation}
  C_n  = \int_{BZ}{\Omega_{\mu\nu}\,ds^\mu\wedge ds^\nu},
\end{equation}
associated to each band by integrating over the Brillouin zone.
The electronic bands of topologically trivial insulators have a Chern number of 0,
and the most famous Chern insulator is the Haldane model of graphene~\cite{Haldane}.
At the interface between topological and conventional insulators the different 
topological invariants force band degeneracy; the adiabatic theorem does not
apply. Subsequently, topological insulators have topologically protected metallic 
surface states because the vacuum is topologically trivial.
Crystals with inversion symmetry have a symmetric Berry curvature
($\Omega_{\mu\nu}(\bm{k}) = \Omega_{\mu\nu}(\bm{-k}) $)
and time-reversal symmetry implies an antisymmetric Berry curvature 
($\Omega_{\mu\nu}(\bm{k}) = -\Omega_{\mu\nu}(\bm{-k}) $). Thus, topologically 
non-trivial insulators cannot posses both time-reversal and inversion symmetry.
Thus far, we have seen Chern numbers used to classify two dimensional 
topological insulators but they can also characterise three dimensional materials
; specifically, Weyl semimetals.

\section{The chiral anomaly in Weyl semimetals}

Weyl semimetals are zero band gap materials where the Hamiltonian near band degeneracies,
at $\bm{k}_w$, takes the form
\begin{equation}
  H(\bm{k}) = \chi v_g \bm{\sigma}\cdot \left(
  \hbar (\bm{k} - \bm{k}_w) - \frac{e}{c}\bm{\mathcal{A}} \right),
\end{equation}
where $\bm{\mathcal{A}}$ is an external magnetic vector potential and
$v_g$ is the pseudoparticle group velocity~\cite{SonSpivak}.
These degeneracies are called Weyl points with chirality
$\chi = \pm$ and are isolated in 
the Brillouin zone because of the von Neumann-Wigner theorem~\cite{AvoidedCrossing}.
Moreover, the Berry curvature is divergence-free across the Brillouin zone
except at points of degeneracy; Weyl points are monopoles of a
`topological charge'
\begin{equation}
  \oint_S{\Omega_{\mu\nu}\,ds^\mu\wedge ds^\nu} = c,
\end{equation}
where the surface $S$ encloses a Weyl point~\cite{SonSpivak}. 
Furthermore, the Nielsen-Ninomiya theorem shows that these Weyl
points come in pairs of opposite chirality~\cite{Nielsen-Ninomiya}.
If the Weyl points are near the Fermi level the fermi surface will consist of electron and
hole states, each surrounding Weyl points.
However, the electron densities surrounding these Weyl nodes at the fermi level are not
always conserved; the chiral anomaly
\begin{equation}
  \frac{dn}{dt} = \frac{e^2}{4\pi^2\hbar^2 c} \bm{E}\cdot\bm{B}.
\end{equation}
This is because the external fields induce a fermion current between Weyl nodes of opposite
chirality that conserves the bulk particle number. This produces a large negative 
longitudinal magnetoresistance in Weyl semimetals that is highly anisotropic. However,
because the Weyl Hamiltonian and dispersion is only approximately true in the vicinity
of Weyl points this may not be observable in condensed matter systems,
as is the case for the chiral magnetic effect~\cite{ChiralMag}. Nevertheless,
the chiral anomaly has been observed in TaAs~\cite{TaAs}.

\vspace*{-5pt}
\bibliography{refs}

\end{document}
